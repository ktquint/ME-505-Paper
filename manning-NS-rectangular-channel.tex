\documentclass[twocolumn, letterpaper]{article}
\usepackage{lipsum}
\usepackage{fancyhdr}
\usepackage{lastpage}
\usepackage{fontspec}
\usepackage{graphicx}
\usepackage{amsmath}
\usepackage{amssymb}
\usepackage{booktabs}
\usepackage[
backend=biber,
style=numeric,
sorting=nty
]{biblatex}
\addbibresource{manning-NS-rectangular-channel.bib}
\graphicspath{ {./} }




\renewcommand{\maketitle}{
    \twocolumn[%
    \raggedleft
    \setmainfont{EuclidFraktur.ttf}
    {\Large Journal of Applied Engineering Mathematics} \\ 
    \vspace{.5 cm}
    \setmainfont{Times New Roman}
    \textbf{Volume 12, December 2025}
    \vspace{1 cm}
    \center
    {\Large \textbf{\thetitle} } \\
    \vspace{.75cm}
    {\large \theauthor } \\
    \vspace{.5cm}

    Civil and Construction Engineering Department \\
    Brigham Young University \\
    Provo, Utah 84602 \\
    {\footnotesize \email}\\
    \vspace{1 cm}
    ]
}



% -------------------------------
% Title, Author, Email
\title{Manning's and Navier-Stokes in Rectangular Channels}
\author{Kenneth Quintana, Anna Cardall}
\newcommand{\email}{ktquint@byu.edu, cardalla@byu.edu}

% ----------------------------------

\makeatletter
\let\thetitle\@title
\let\theauthor\@author
\makeatother

\begin{document}
\setmainfont{Times New Roman}

\pagestyle{fancy}
\lhead{\theauthor. \thetitle  %\thepage - \pageref{LastPage} 
}
\rhead{ / JAEM 12 (2025) \thepage\ of~\pageref{LastPage} }
\lfoot{\scriptsize \textbf{Journal of Applied Engineering Mathematics} December 2025, Vol. 12}
\rfoot{\footnotesize Copyright \copyright 2025 by ME505 BYU}


\maketitle
% Add your stuff starting from here, aside from the title, name, and email above
\begin{abstract}
    % Abstract here
    Manning's equation is widely used for estimating flow in open channels due
    to its simplicity and empirical basis. However, its accuracy can be limited
    in complex flow conditions. This study compares the performance of Manning's
    equation with the Navier-Stokes equations in rectangular channels under
    varying Reynold's Numbers. Assuming a smooth channle ($n$ = 0.01 to 0.012),
    we compared the two flowrates. Our results indicated that exact matches between the two
    equations occurent when the aspect ratio ($\alpha$) of the channel was between 17 and 45,
    with errors less than 5\%. This agreement occurred when the flow depth was significantly
    greater than the channel width, suggesting that under these conditions, the roughness
    assumptions of Manning's equation align more closely with the fundamental physics
    captured by the Navier-Stokes equations. These findings highlight the limitations of
    Manning's equation in laminar flow conditions and those of applying the simplified
    Navier-Stokes equations to turbulent flow.

\end{abstract}

\section*{Nomenclature}
\begin{description}
\item[Navier-Stokes Variables:]
\item[$u_{i/j}$] = Velocity vector (m/s)  
\item[{$P$}] = Pressure (Pa)
\item[$\mu$] = Dynamic viscosity (Pa·s)
\item[$g_i$] = Gravitational acceleration vector (m/s$^2$)
\item[$t$] = Time (s)
\item[$x_{i/j}$] = Spatial coordinate (m)
\item[$\rho$] = Fluid density (kg/m$^3$)
\item[$\nu$] = Kinematic viscosity (m$^2$/s)
\item[Manning's Equation Variables:]
\item[$Q$] = Volumetric Flowrate (m$^3$/s)
\item[$A$] = Cross-sectional area of flow (m$^2$)
\item[$n$] = Manning's roughness coefficient
\item[$R_H$] = Hydraulic radius (m)
\item[$P_w$] = Wetted perimeter (m)
\item[$S$] = Channel slope (m/m)
\item[$V$] = Average velocity (m/s)
\item[Shared Variables:] 
\item[$b$] = Channel bottom width (m)
\item[$h$] = Flow depth (m)
\item[$Re$] = Reynolds number (dimensionless)
\end{description}

\section*{Introduction}
Measuring and predicting flow in open channels is a fundamental aspect of both
hydraulic engineering and engineering hydrology. Several methods exist for estimating
flow characteristics, the most import of which being volumetric flow rate ($Q$).

The most commonly used method for estimating flow in open channels is Manning's equation:

\begin{equation}
    Q = \frac{1}{n} A R_H^{2/3} S^{1/2},
\end{equation}

where hydraulic radius $R_H$ is defined as the ratio of the cross-sectional area of flow to
the wetted perimeter, given as

\begin{equation}
    R_H = \frac{A}{P_w}.
\end{equation}

Manning's equation is applied to all open channels, from culverts, streams, and rivers to
large canals with gravity-driven flows \cite{GioiaPhysRevLett}. It's popularity stems from its simplicity and 
empirical basis, allowing engineers to quickly estimate flow rates based on channel 
characteristics \cite{GioiaPhysRevLett}. However, Manning's equation has several limitations, 
including its purely empirical (rather than theoretical) nature, assumptions of uniform flow,
and sensitivity to the roughness coefficient, $n$, which must be estimated for the channel 
\cite{DINGMAN199713}. These limitations can lead to inaccuracies in flow conditions that
exhibit varying channel geometries, varying channel roughnesses, unsteady flows, and laminar 
flow regimes.

For more complex flow conditions, the Navier-Stokes equations provide a comprehensive
framework for modeling fluid dynamics. The Navier-Stokes equations describe the motion
of viscous fluid substances and are derived from the principles of conservation of
mass, momentum, and energy. They can capture a wide range of flow phenomena,
including turbulence, boundary layer effects, and non-uniform flow profiles.

In this paper, we evaluate the accuracy of Manning's equations for laminar flow conditions
in an open rectangular channel (\ref{fig:channel-and-coordinates}). We do this by comparing
volumetric flowrates obtained with Navier-Stokes equation for laminar flow to those obtained 
by Manning's equation for several flow velocities and channel aspect ratios. We do not solve 
the full Navier-Stokes equations, but rather simplify them using the conditions for Manning's 
equation, except for turbulence.

\begin{figure} 
    \centering
    \includegraphics[width=.45\textwidth]{figures/channel-and-coordinates 1.png}
    \caption{Rectangular channel and coordinate system.}
    \label{fig:channel-and-coordinates}
\end{figure}

\section*{Methodology}

\subsection*{Governing Equations}
We determine the flow regime using the Reynolds number ($Re$), defined as

\begin{equation}
    Re = \frac{V R_H}{\nu},
\end{equation}

where $V$ is the characteristic velocity (m/s), $R_H$ is the hydraulic
radius (m), and $\nu$ is the kinematic viscosity (m$^2$/s). We will consider laminar flow conditions to be such that $Re < 500$.

For gravity-driven flows, the Navier Stokes equation is given as

\begin{equation}    
   \rho \left( \frac{\partial u_i}{\partial t} + u_j \frac{\partial u_i}{\partial x_j} \right) = -\frac{\partial P}{\partial x_i} + \mu \frac{\partial^2 u_i}{\partial x_j^2} + \rho g_i,
\end{equation}

where $\rho$ is the fluid density (kg/m$^3$), $u_i$ is the velocity vector (m/s), $P$
is the pressure (Pa), $\mu$ is the dynamic viscosity (Pa·s), and $g_i$ is the
gravitational acceleration vector (m/s$^2$).


\subsection*{Assumptions}
The following assumptions were made to simplify the Navier-Stokes to match the conditions necessary to apply Manning's equation:
\begin{itemize}
    \item The fluid is water $\rightarrow \rho = C, \mu = C$.
    \item Flow is steady $\rightarrow \frac{\partial}{\partial t} = 0$.
    \item Flow is uniform $\rightarrow \frac{\partial P}{\partial x} = 0$.
    \item Channel is infinite in the flow direction $\rightarrow \frac{\partial}{\partial x} = 0$.
    \item Flow is irrotational $\longrightarrow \mathbf{u} \cdot \nabla \mathbf{u} = 0$.
\end{itemize}

Applying our assumptions to the x-momentum equation, we simplify it to
\begin{equation}
    -\rho g_x = \mu \left(\frac{\partial^2 u}{\partial y^2} + \frac{\partial^2 u}{\partial z^2}\right),
\end{equation}
where $g_x$ is the component of gravitational acceleration in the x-direction (m/s$^2$), and $u$ is the velocity component in the x-direction (m/s).


With boundary conditions:
\begin{itemize}
    \item No slip at the bottom and sides: $u = 0$ at $y = 0$, $z = 0$, and $z = b$.
    \item Symmetry at the free surface: $\frac{\partial u}{\partial y} = 0$ at $y = h$.
\end{itemize}

\begin{figure} 
    \centering
    \includegraphics[width=.45\textwidth]{figures/boundary-conditions 1.png}
    \caption{Boundary conditions for open channel flow.}
    \label{fig:channel-boundary-conditions}
\end{figure}


\subsection*{Velocity and Flowrate Calculations}

Next we non-dimensionalize the equations using the following variables:
\begin{equation}
    \hat{y} = \frac{y}{h}, \quad \hat{z} = \frac{z}{b}, \quad \hat{u} = \frac{u \nu}{h^2 g \sin \theta},
\end{equation}
where $\theta$ is the angle of the channel slope.
Substituting these into the simplified Navier-Stokes equation, we obtain the non-dimensional form
\begin{equation}
    -1 = \frac{\partial^2 \hat{u}}{\partial \hat{y}^2} + \alpha^2 \frac{\partial^2 \hat{u}}{\partial \hat{z}^2},
\end{equation}
where $\alpha = \frac{h}{b}$ is the aspect ratio of the channel.

We solve this equation using separation of variables and apply the boundary conditions to find the velocity profile
\begin{equation}
    \hat{u}(\hat{y}, \hat{z}) = \hat{y} - \frac{\hat{y}^2}{2} + \sum_{n=1}^{\infty}A_n\sin(\lambda_n \hat{y})\cosh\left[\frac{\lambda_n}{\alpha}\left(\hat{z}-\frac{1}{2}\right)\right],
\end{equation}

where $A_n = \frac{-2}{\lambda_n^3 \cosh\left(\frac{\lambda_n}{2\alpha}\right)}$.

As shown in Figure~\ref{fig:3d-profile}, the flow distribution meets the boundary conditions defined previously.

\begin{figure}
    \centering
    \includegraphics[width=0.45\textwidth]{figures/3d-profile.png}
    \caption{Non-dimensionalized velocity profile.}
    \label{fig:3d-profile}
\end{figure}

The volumetric flow rate $Q$ is calculated by integrating the velocity profile over the cross-sectional area:
\begin{equation}
    Q = \int_0^b \int_0^h u(y, z) \, dy \, dz.
\end{equation}

We are left with a final expression for $Q$ in terms of channel dimensions and fluid properties:

\begin{equation}
    Q = \frac{h^3 b g \sin \theta}{\nu} \left[ \frac{1}{3} - \frac{4h}{b} \sum_{n=1}^{\infty} \frac{1}{\lambda_n^5} \tanh\left(\frac{\lambda_n b}{2h}\right) \right].
\end{equation}


Manning's equation for volumetric flow rate in a rectangular channel is given by
\begin{equation}
    Q = \frac{1}{n} A R_H^{2/3} S^{1/2}.
\end{equation}
Substituting the expressions for $A$ and $R_H$ for a rectangular channel and rearranging
to solve for $n$, we have

\begin{equation}
    n = \frac{1}{Q}(b h) {\left(\frac{b h}{b + 2h}\right)}^{2/3} S^{1/2}.
\end{equation}

\subsection*{Reynold's Number Selection}
To evaluate the performance of Manning's equation under laminar flow conditions,
we selected a range of Reynolds numbers ($Re$) from 250 (laminar) to 12,500
(fully turbulent).

For each selected $Re$, we calculated the flowrate for a combination of channel
widths ($b$ = 0.01 to 5 m) and slopes ($S$ = 0.01 to 0.1 m/m). From the 
Navier-Stokes derived flowrate, we calculated Manning's $n$ using the rearranged
Manning's equation. If Manning's equation were accurate, the calculated $n$
values would fall within typical ranges for smooth rectangular channels
(0.009 to 0.012).

\section*{Results and Discussion}

We found that no Reynolds numbers yielded similar discharge values in both
the Navier-Stokes and Manning's equations. when using typical values of Manning's $n$ for
smooth rectangular channels. Figure~\ref{fig:flowrate-agreement} shows the comparison of
flowrates across tested Reynolds numbers, with small regions where the two flowrates are
equivalent. Table~\ref{tab:reynolds_analysis} summarizes the best
matches found for each Reynolds number, along with the corresponding aspect ratio

\begin{figure}
    \centering
    \includegraphics[width=0.45\textwidth]{figures/flowrate-agreement.png}
    \caption{Comparison of Navier-Stokes and Manning's flowrates across Reynolds numbers.}
    \label{fig:flowrate-agreement}
\end{figure}


\begin{table}[h!]
    \centering
    \caption{Best Match per Reynolds Number}
    \label{tab:reynolds_analysis}
    \begin{tabular}{c c c}
        \toprule
        \textbf{Re} & \textbf{Optimal $\boldsymbol{\alpha}$} & \textbf{Error (\%)} \\
        \midrule
        250   & 34.1 & 0.0544 \\
        500   & 24.7 & 0.0167 \\
        2,000  & 23.2 & 0.593 \\
        12,500 & 24.1 & 0.0509 \\
        \bottomrule
    \end{tabular}
\end{table}

Exploring further, we found a range of aspect ratios where the percent error between was minimized.
Figure~\ref{fig:error-analysis} shows the error between the two flowrates across aspect ratios for
all tested Reynolds numbers. The area highlighted in green indicates the range where the comparisons have
an error of less than 5\%.


\begin{figure}
    \centering
    \includegraphics[width=0.45\textwidth]{figures/error-analysis-range.png}
    \caption{Error reduction of Navier-Stokes and Manning's Discharge.}
    \label{fig:error-analysis}
\end{figure}

Our analysis revealed that across all Reynolds numbers the Navier-Stokes flowrate was greater
than the Manning's flowrate for aspect ratios greater than about 45 The opposite was true for
aspect ratios less than about 17. Table~\ref{tab:global_alpha_range}
summarizes the global aspect ratio range where the error between the two flowrates is
less than 5\%.

\begin{table}[h!]
    \centering
    \caption{Global $\alpha$ Range with $< 5.0\%$ Error}
    \label{tab:global_alpha_range}
    \begin{tabular}{l c}
        \toprule
        \textbf{Statistic} & \textbf{Value} \\
        \midrule
        Min $\alpha$  & 17.0 \\
        Max $\alpha$  & 44.8 \\
        Mean $\alpha$ & 26.7 \\
        \bottomrule
    \end{tabular}
\end{table}

The alpha values minimizing percent error suggest agreement occurs when the depth
of flow is much greater than the width of the channel. Considering the fundamental physics
governing the Navier-Stokes equations and the empirical nature of Manning's, we understanding
why we see some agreement. Navier-Stokes account for the viscous forces acting
across the entire cross-section of the flow, while Manning's simplifies these
effects into a single roughness coefficient $n$. When the flow depth is significantly
greater than the channel width, the influence of roughness on the channel walls is felt 
throughout the entire flow area. With roughness affecting the entire flow, the friction forces
from Manning's equation behave more similarly to the viscous forces from the Navier-Stokes,
leading to closer agreement in flowrates.

\section*{Conclusions}
Our study compared the performance of Manning's equation with the Navier-Stokes equations
in rectangular channels under laminar and turbulent flow conditions. We compared an analytically 
derived flowrate from the Navier-Stokes equations to the empirically derived flowrate
from Manning's equation across a range of Reynolds numbers (250 to 12,500). Our results indicated
that exact matches between the two flowrates were not achievable across any Reynolds numbers using typical
values of Manning's $n$ for smooth channels. However, we identified a range of aspect ratios
(17 to 45) where the error between the two flowrates was less than 5\%. Agreement occured
when the flow depth was significantly greater than the channel width, suggesting that the roughness assumptions
of Manning's equation align more closely with the viscous forces captured by Navier-Stokes. These findings
highlight the limitations of Manning's equation in laminar flow conditions and those of applying the simplified
Navier-Stokes equations to turbulent flow.

\section*{Acknowledgements}
The authors would like to acknowledge Dr.\ Vladimir P. Soloviev for his guidance and support in teaching the ME 505 course on Applied Engineering Mathematics.
His patience and expertise were invaluable in guiding our understanding the fundamental concepts of differential equations and the solution methods we employed in this paper.

% The bibliography is automatically generated from manning-NS-rectangular-channel.bib. Rename template.bib to match this file's name, if you change it. This will require an installation of Biber, which you should not install by hand; you run Latex on this document, then Biber on the shared name (e.g. "template"), then Latex again.
\printbibliography{}

\section*{Appendix}
Put appendix here.
\end{document}
