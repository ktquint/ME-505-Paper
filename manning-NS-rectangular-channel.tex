\documentclass[twocolumn, letterpaper]{article}
\usepackage{lipsum}
\usepackage{fancyhdr}
\usepackage{lastpage}
\usepackage{fontspec}
\usepackage{graphicx}
\usepackage{amsmath}
\usepackage{amssymb}
\usepackage[
backend=biber,
style=numeric,
sorting=nty
]{biblatex}
\addbibresource{manning-NS-rectangular-channel.bib}
\graphicspath{ {./} }




\renewcommand{\maketitle}{
    \twocolumn[%
    \raggedleft
    \setmainfont{EuclidFraktur.ttf}
    {\Large Journal of Applied Engineering Mathematics} \\ 
    \vspace{.5 cm}
    \setmainfont{Times New Roman}
    \textbf{Volume 12, December 2025}
    \vspace{1 cm}
    \center
    {\Large \textbf{\thetitle} } \\
    \vspace{.75cm}
    {\large \theauthor } \\
    \vspace{.5cm}

    Civil and Construction Engineering Department \\
    Brigham Young University \\
    Provo, Utah 84602 \\
    {\footnotesize \email}\\
    \vspace{1 cm}
    ]
}



% -------------------------------
% Title, Author, Email
\title{Manning's and Navier-Stokes in Rectangular Channels}
\author{Kenneth Quintana, Anna Cardall}
\newcommand{\email}{ktquint@byu.edu, cardalla@byu.edu}

% ----------------------------------

\makeatletter
\let\thetitle\@title
\let\theauthor\@author
\makeatother

\begin{document}
\setmainfont{Times New Roman}

\pagestyle{fancy}
\lhead{\theauthor. \thetitle  %\thepage - \pageref{LastPage} 
}
\rhead{ / JAEM 12 (2025) \thepage\ of~\pageref{LastPage} }
\lfoot{\scriptsize \textbf{Journal of Applied Engineering Mathematics} December 2025, Vol. 12}
\rfoot{\footnotesize Copyright \copyright 2025 by ME505 BYU}


\maketitle
% Add your stuff starting from here, aside from the title, name, and email above
\begin{abstract}
    % Abstract here
    Manning's equation is widely used for estimating flow in open channels due
    to its simplicity and empirical basis. However, its accuracy can be limited
    in complex flow conditions. This study compares the performance of Manning's
    equation with the Navier-Stokes equations in rectangular channels under
    varying Reynold's Numbers. Assuming laminar flow ($Re$ < 500), we calculate 
    values of Manning's $n$ and compare them to traditional values.
    Our results indicate

\end{abstract}

\section*{Nomenclature}
\begin{description}
\item[Navier-Stokes Variables:]
\item[$u_{i/j}$] = Velocity vector (m/s)  
\item[{$P$}] = Pressure (Pa)
\item[$\mu$] = Dynamic viscosity (Pa·s)
\item[$g_i$] = Gravitational acceleration vector (m/s$^2$)
\item[$t$] = Time (s)
\item[$x_{i/j}$] = Spatial coordinate (m)
\item[$\rho$] = Fluid density (kg/m$^3$)
\item[$\nu$] = Kinematic viscosity (m$^2$/s)
\item[Manning's Equation Variables:]
\item[$Q$] = Volumetric Flowrate (m$^3$/s)
\item[$A$] = Cross-sectional area of flow (m$^2$)
\item[$n$] = Manning's roughness coefficient
\item[$R_H$] = Hydraulic radius (m)
\item[$P_w$] = Wetted perimeter (m)
\item[$S$] = Channel slope (m/m)
\item[$V$] = Average velocity (m/s)
\item[Shared Variables:] 
\item[$b$] = Channel bottom width (m)
\item[$h$] = Flow depth (m)
\item[$Re$] = Reynolds number (dimensionless)
\end{description}

\section*{Introduction}
Measuring and predicting flow in open channels is a fundamental aspect of both
hydraulic engineering and engineering hydrology. Several methods exist for estimating
flow characteristics, the most import of which being volumetric flow rate ($Q$).

The most commonly used method for estimating flow in open channels is Manning's equation:

\begin{equation}
    Q = \frac{1}{n} A R_H^{2/3} S^{1/2}
\end{equation}

Where hydraulic radius $R_H$ is defined as the ratio of the cross-sectional area of flow to
the wetted perimeter:

\begin{equation}
    R_H = \frac{A}{P_w}
\end{equation}

Manning's equation is applied to all open channels, from culverts, streams, and rivers to
large canals \cite{GioiaPhysRevLett}. It's popularity stems from its simplicity and empirical basis, allowing
engineers to quickly estimate flow rates with limited data. However, Manning's equations
has several limitations, including its empirical nature, assumptions of uniform flow,
and sensitivity to the roughness coefficient, $n$. These limitations can lead to inaccuracies
in complex flow conditions, such as varying channel geometries, unsteady flows, and turbulent
regimes.

For more complex flow conditions, the Navier-Stokes equations provide a comprehensive
framework for modeling fluid dynamics. The Navier-Stokes equations describe the motion
of viscous fluid substances and are derived from the principles of conservation of
mass, momentum, and energy. They can capture a wide range of flow phenomena,
including turbulence, boundary layer effects, and non-uniform flow profiles.

In this paper, we will not solve the full Navier-Stokes equations, but rather simplify
them to matching conditions necessary to apply Manning's equation, except for turbulence.
Our solutions to the Navier-Stokes equations will yield an analytical expression for
volumetric flowrate under laminar flow conditions in rectangular channels. We will then
compare the analytically derived flowrates to the empirically derived flowrates from
Manning's equation to evaluate its accuracy under laminar flow conditions.

We will determine the flow regime using the Reynolds number ($Re$), defined as:

\begin{equation}
    Re = \frac{V R_H}{\nu}
\end{equation}

where $V$ is the characteristic velocity (m/s), $R_H$ is the hydraulic
radius (m), and $\nu$ is the kinematic viscosity (m$^2$/s). We will consider laminar flow conditions to be such that $Re < 500$.

\begin{figure} 
    \centering
    \includegraphics[width=.45\textwidth]{figures/channel-and-coordinates 1.png}
    \caption{Rectangular channel and coordinate system.}
    \label{fig:channel-and-coordinates}
\end{figure}

\section*{Methodology}

\subsection*{Governing Equations}

We start our derivation with the Navier-Stokes equations in three dimensions:
\begin{equation}    
   \rho \left( \frac{\partial u_i}{\partial t} + u_j \frac{\partial u_i}{\partial x_j} \right) = -\frac{\partial P}{\partial x_i} + \mu \frac{\partial^2 u_i}{\partial x_j^2} + \rho g_i
\end{equation}
where $\rho$ is the fluid density (kg/m$^3$), $u_i$ is the velocity vector (m/s), $P$
is the pressure (Pa), $\mu$ is the dynamic viscosity (Pa·s), and $g_i$ is the
gravitational acceleration vector (m/s$^2$).


\subsection*{Assumptions}
The following assumptions were made to simplify the Navier-Stokes to match the conditions necessary to apply Manning's equation:
\begin{itemize}
    \item The fluid is water $\rightarrow \rho = C, \mu = C$.
    \item Flow is steady $\rightarrow \frac{\partial}{\partial t} = 0$.
    \item Flow is uniform $\rightarrow \frac{\partial P}{\partial x} = 0$.
    \item Channel is infinite in the flow direction $\rightarrow \frac{\partial}{\partial x} = 0$.
    \item Flow is irrotational $\longrightarrow \mathbf{u} \cdot \nabla \mathbf{u} = 0$.
\end{itemize}

Applying our assumptions to the x-momentum equation, we simplify it to:
\begin{equation}
    -\rho g_x = \mu \left(\frac{\partial^2 u}{\partial y^2} + \frac{\partial^2 u}{\partial z^2}\right)
\end{equation}
where $g_x$ is the component of gravitational acceleration in the x-direction (m/s$^2$), and $u$ is the velocity component in the x-direction (m/s).


With boundary conditions:
\begin{itemize}
    \item No slip at the bottom and sides: $u = 0$ at $y = 0$, $z = 0$, and $z = b$.
    \item Symmetry at the free surface: $\frac{\partial u}{\partial y} = 0$ at $y = h$.
\end{itemize}

\begin{figure} 
    \centering
    \includegraphics[width=.45\textwidth]{figures/boundary-conditions 1.png}
    \caption{Boundary conditions for open channel flow.}
    \label{fig:channel-boundary-conditions}
\end{figure}


\subsection*{Velocity and Flowrate Calculations}

Next we non-dimensionalize the equations using the following variables:
\begin{equation}
    \hat{y} = \frac{y}{h}, \quad \hat{z} = \frac{z}{b}, \quad \hat{u} = \frac{u \nu}{h^2 g \sin \theta}
\end{equation}
where $\theta$ is the angle of the channel slope.
Substituting these into the simplified Navier-Stokes equation, we obtain the non-dimensional form:
\begin{equation}
    -1 = \frac{\partial^2 \hat{u}}{\partial \hat{y}^2} + \alpha^2 \frac{\partial^2 \hat{u}}{\partial \hat{z}^2}
\end{equation}
where $\alpha = \frac{h}{b}$ is the aspect ratio of the channel.

We solve this equation using separation of variables and apply the boundary conditions to find the velocity profile
\begin{equation}
    \hat{u}(\hat{y}, \hat{z}) = \hat{y} - \frac{\hat{y}^2}{2} + \sum_{n=1}^{\infty}A_n\sin(\lambda_n \hat{y})\cosh\left[\frac{\lambda_n}{\alpha}\left(\hat{z}-\frac{1}{2}\right)\right]
\end{equation}

where $A_n = \frac{-2}{\lambda_n^3 \cosh\left(\frac{\lambda_n}{2\alpha}\right)}$.



As shown in Figure~\ref{fig:3d-profile}, the flow distribution meets the boundary conditions defined previously.

\begin{figure}
    \centering
    \includegraphics[width=0.45\textwidth]{figures/3d-profile.png}
    \caption{Non-dimensionalized velocity profile.}
    \label{fig:3d-profile}
\end{figure}

The volumetric flow rate $Q$ is calculated by integrating the velocity profile over the cross-sectional area:
\begin{equation}
    Q = \int_0^b \int_0^h u(y, z) \, dy \, dz
\end{equation}

We are left with a final expression for $Q$ in terms of channel dimensions and fluid properties:

\begin{equation}
    Q = \frac{h^3 b g \sin \theta}{\nu} \left[ \frac{1}{3} - \frac{4h}{b} \sum_{n=1}^{\infty} \frac{1}{\lambda_n^5} \tanh\left(\frac{\lambda_n b}{2h}\right) \right]
\end{equation}


Manning's equation for volumetric flow rate in a rectangular channel is given by:
\begin{equation}
    Q = \frac{1}{n} A R_H^{2/3} S^{1/2}
\end{equation}
Substituting the expressions for $A$ and $R_H$ for a rectangular channel and rearranging
to solve for $n$, we have:

\begin{equation}
    n = \frac{1}{Q}(b h) {\left(\frac{b h}{b + 2h}\right)}^{2/3} S^{1/2}
\end{equation}

\subsection*{Reynold's Number Selection}
To evaluate the performance of Manning's equation under laminar flow conditions,
we selected a range of Reynolds numbers ($Re$) from 250 (laminar) to 12,500
(fully turbulent).

$Re$ is defined as:
\begin{equation}
    Re = \frac{V R_H}{\nu}
\end{equation}
where $V$ is the average velocity (m/s).

For each selected $Re$, we calculated the flowrate for a combination of channel
widths ($b$ = 0.01 to 5 m) and slopes ($S$ = 0.01 to 0.1 m/m). From the 
Navier-Stokes derived flowrate, we calculated Manning's $n$ using the rearranged
Manning's equation. If Manning's equation were accurate, the calculated $n$
values would fall within typical ranges for smooth rectangular channels
(0.009 to 0.012).

\section*{Results and Discussion}

\begin{figure} 
    \centering
    \includegraphics[width=.45\textwidth]{figures/n-vs-Q-Re.png}
    \caption{Manning's $n$ vs. Discharge for several values of $Re$.}
    \label{fig:n-vs-Q}
\end{figure}

\section*{Conclusions}
Put conclusions here.

\section*{Acknowledgements}
The authors would like to acknowledge Dr.\ Vladimir P. Soloviev for his guidance and support in teaching the ME 505 course on Applied Engineering Mathematics.
His patience and expertise were invaluable in guiding our understanding the fundamental concepts of differential equations and the solution methods we employed in this paper.
The authors would also like to thank Dr.\ Julie Crockett for her assistance in understanding the fundamental principles of physics used to derive the Navier-Stokes equations.
Her teachings inspired the initial research question that led to this study. 

% The bibliography is automatically generated from manning-NS-rectangular-channel.bib. Rename template.bib to match this file's name, if you change it. This will require an installation of Biber, which you should not install by hand; you run Latex on this document, then Biber on the shared name (e.g. "template"), then Latex again.
\printbibliography{}

\section*{Appendix}
Put appendix here.
\end{document}
