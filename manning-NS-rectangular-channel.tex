% !TEX program = xelatex
\documentclass[twocolumn, letterpaper]{article}
\usepackage{lipsum}
\usepackage{fancyhdr}
\usepackage{lastpage}
\usepackage{fontspec}
\usepackage{graphicx}
\usepackage{amsmath}
\usepackage{amssymb}
\usepackage[
backend=biber,
style=numeric,
sorting=nty
]{biblatex}
\addbibresource{template.bib}
\graphicspath{ {./} }




\renewcommand{\maketitle}{
    \twocolumn[%
    \raggedleft
    \setmainfont{EuclidFraktur.ttf}
    {\Large Journal of Applied Engineering Mathematics} \\ 
    \vspace{.5 cm}
    \setmainfont{Times New Roman}
    \textbf{Volume 12, December 2025}
    \vspace{1 cm}
    \center
    {\Large \textbf{\thetitle} } \\
    \vspace{.75cm}
    {\large \theauthor } \\
    \vspace{.5cm}

    Civil and Construction Engineering Department \\
    Brigham Young University \\
    Provo, Utah 84602 \\
    {\footnotesize \email}\\
    \vspace{1 cm}
    ]
}

% -------------------------------
% Title, Author, Email
\title{Comparison of Manning and Navier-Stokes in Rectangular Channels}
\author{Kenneth T. Quintana}
\newcommand{\email}{ktquint@byu.edu}

% ----------------------------------

\makeatletter
\let\thetitle\@title
\let\theauthor\@author
\makeatother

\begin{document}
\setmainfont{Times New Roman}

\pagestyle{fancy}
\lhead{\theauthor. \thetitle  %\thepage - \pageref{LastPage} 
}
\rhead{ / JAEM 12 (2025) \thepage\ of \pageref{LastPage} }
\lfoot{\scriptsize \textbf{Journal of Applied Engineering Mathematics} December 2025, Vol. 12}
\rfoot{\footnotesize Copyright \copyright 2025 by ME505 BYU}


\maketitle

% Add your stuff starting from here, aside from the title, name, and email above
\begin{abstract}
    % Abstract here
    Manning's equation is widely used for estimating flow in open channels due
    to its simplicity and empirical basis. However, its accuracy can be limited
    in complex flow conditions. This study compares the performance of Manning's
    equation with the Navier-Stokes equations in rectangular channels under
    varying Reynold's Numbers. Assuming laminar flow ($Re$ < 500), we calculate 
    values of Manning's $n$ and compare them to traditional values.
    Our results indicate that for laminar flow, Manning's equation provides
    reasonable estimates for uniform flow. Under turbulent and transitional flows,
    the Navier-Stokes equations do not hold, so we made no comparison. The findings
    highlight the importance of selecting appropriate modeling techniques based on
    flow conditions for accurate hydraulic predictions.
\end{abstract}

\section*{Nomenclature}
$Q$ = Discharge (m$^3$/s) \\
$A$ = Cross-sectional area of flow (m$^2$) \\
$n$ = Manning's roughness coefficient \\
$R_H$ = Hydraulic radius (m) \\
$P_w$ = Wetted perimeter (m) \\
$S$ = Channel slope (m/m) \\
$V$ = Velocity (m/s) \\
$b$ = Channel bottom width (m) \\
$h$ = Flow depth (m) \\
$\nu$ = Kinematic viscosity (m$^2$/s) \\
$g$ = Gravitational acceleration (m/s$^2$) \\
$\rho$ = Fluid density (kg/m$^3$)\\
$Re$ = Reynolds number (dimensionless) \\

\section*{Introduction}
Manning's equation is an empirical formula used to estimate the flow of water in open channels. It is expressed as:

\begin{equation}
    Q = \frac{1}{n} A R_H^{2/3} S^{1/2}
\end{equation}

where $Q$ is the discharge (m$^3$/s), $A$ is the cross-sectional area of flow (m$^2$), $R_H$ is the hydraulic radius (m), $S$ is the channel slope (m/m), and $n$ is Manning's roughness coefficient, which accounts for the channel's surface roughness.
The hydraulic radius $R_H$ is defined as the ratio of the cross-sectional area of flow to the wetted perimeter:
\begin{equation}
    R_H = \frac{A}{P_w}
\end{equation}
For a rectangular channel, the cross-sectional area $A$ and wetted perimeter $P_w$ can be expressed as:
\begin{equation}
    A = b \cdot h
\end{equation}
\begin{equation}
    P_w = b + 2h
\end{equation}
where $b$ is the channel bottom width (m) and $h$ is the flow depth (m).

The Navier-Stokes equations describe the motion of fluid substances and are fundamental to fluid dynamics. For incompressible flow, the Navier-Stokes equations can be written as:
\begin{equation}    
    \rho \left( \frac{\partial \mathbf{u}}{\partial t} + \mathbf{u} \cdot \nabla \mathbf{u} \right) = -\nabla p + \mu \nabla^2 \mathbf{u} + \rho \mathbf{g}
\end{equation}
where $\rho$ is the fluid density (kg/m$^3$), $\mathbf{u}$ is the velocity vector (m/s), $p$ is the pressure (Pa), $\mu$ is the dynamic viscosity (Pa·s), and $\mathbf{g}$ is the gravitational acceleration vector (m/s$^2$).
The Reynolds number ($Re$) is a dimensionless quantity used to predict flow patterns in different fluid flow situations. It is defined as:
\begin{equation}
    Re = \frac{V R_h}{\nu}
\end{equation}
where $V$ is the characteristic velocity (m/s), $R_H$ is the hydraulic
radius (m), and $\nu$ is the kinematic viscosity (m$^2$/s). For a rectangular channel, the hydraulic diameter can be expressed as:
\begin{equation}
    R_H = \frac{A}{P_w}
\end{equation}  
In this study, we compare the performance of Manning's equation with the Navier-Stokes equations in rectangular channels under varying Reynolds numbers. We focus on laminar flow conditions ($Re$ < 500) to evaluate the accuracy of Manning's equation in predicting flow characteristics.


\section*{Assumptions}
The following assumptions are made in this study:
\begin{itemize}
    \item The flow is steady $\longrightarrow \frac{\partial}{\partial t} = 0$.
    \item The flow is uniform $\longrightarrow \frac{\partial P}{\partial x} = 0$.
    \item The channel is infinite in the direction of flow $\longrightarrow \frac{\partial}{\partial x} = 0$.
    \item The fluid is Newtonian.
    \item The fluid is incompressible and irrotational $\longrightarrow \mathbf{u} \cdot \nabla \mathbf{u} = 0$.
    \item The flow is laminar ($Re$ < 500).
    \item The effects of turbulence and secondary flows are neglected.
\end{itemize}

\section*{Methodology}
Applying our assumptions to the Navier-Stokes equations in the x-direction, we simplify it to:
\begin{equation}
    -\rho g_x = \mu [\frac{\partial^2 u}{\partial y^2} + \frac{\partial^2 u}{\partial z^2}]
\end{equation}
where $g_x$ is the component of gravitational acceleration in the x-direction (m/s$^2$), and $u$ is the velocity component in the x-direction (m/s).
With boundary conditions:
\begin{itemize}
    \item No slip at the bottom and sides: $u = 0$ at $y = 0$, $z = 0$, and $z = b$.
    \item Symmetry at the free surface: $\frac{\partial u}{\partial y} = 0$ at $y = h$.
\end{itemize}

Next we non-dimensionalize the equations using the following variables:
\begin{equation}
    \hat{y} = \frac{y}{h}, \quad \hat{z} = \frac{z}{b}, \quad \hat{u} = \frac{u \nu}{h^2 g \sin \theta}
\end{equation}
where $\theta$ is the angle of the channel slope.
Substituting these into the simplified Navier-Stokes equation, we obtain the non-dimensional form:
\begin{equation}
    -1 = \frac{\partial^2 \hat{u}}{\partial \hat{y}^2} + \left( \frac{h^2}{b^2} \right) \frac{\partial^2 \hat{u}}{\partial \hat{z}^2}
\end{equation}
We solve this equation using separation of variables and apply the boundary conditions to find the velocity profile $\hat{u}(\hat{y}, \hat{z})$.
Once we have the velocity profile, we calculate the volumetric flow rate $Q$ by integrating the velocity over the cross-sectional area:
\begin{equation}
    Q = \int_0^b \int_0^h u(y, z) \, dy \, dz
\end{equation}
Finally, we compare the flow rate obtained from the Navier-Stokes solution with that predicted by Manning's equation to evaluate its accuracy under laminar flow conditions.

\section*{Results and Discussion}
Substituting the non-dimensionalized variables into the boundary conditions, they become:
\begin{itemize}
    \item No slip at the bottom and sides: $\hat{u} = 0$ at $\hat{y} = 0$ and $\hat{z} = 0$, $\hat{z} = 1$.
    \item Symmetry at the free surface: $\frac{\partial \hat{u}}{\partial \hat{y}} = 0$ at $\hat{y} = 1$.
\end{itemize}

We can now solve for $\hat{u}$ as $\hat{u} = u_1(\hat{y}) + u_2(\hat{y}, \hat{z})$, where $u_1$ is the particular solution and $u_2$ is the homogeneous solution.

The equation for the particular solution, $u_1$, is:
\begin{equation}
    \frac{d^2 u_1}{d \hat{y}^2} = -1
\end{equation}
Integrating twice and applying the boundary conditions, we find:
\begin{equation}
    u_1(\hat{y}) = \hat{y} - \frac{\hat{y}^2}{2}
\end{equation} 
The homogeneous solution, $u_2$, satisfies:
\begin{equation}
    \frac{\partial^2 u_2}{\partial \hat{y}^2} + \alpha^2 \frac{\partial^2 u_2}{\partial \hat{z}^2} = 0
\end{equation}
where $\alpha = \frac{h}{b}$.
Using separation of variables, we assume $u_2(\hat{y}, \hat{z}) = Y(\hat{y}) Z(\hat{z})$. Substituting into the homogeneous equation and separating variables, we obtain:
\begin{equation}
    \frac{Y''}{Y} = -\lambda^2, \quad \frac{Z''}{Z} = \alpha^2 \lambda^2
\end{equation}
where $\lambda^2$ is the separation constant.
Solving these ordinary differential equations, we find:
\begin{equation}
    Y(\hat{y}) = A \cos\left(\lambda \hat{y} \right) + B \sin\left( \lambda \hat{y} \right)
\end{equation}
\begin{equation}
    Z(\hat{z}) = C \cosh(\frac{\lambda \hat{z}}{\alpha}) + D \sinh(\frac{\lambda \hat{z}}{\alpha})
\end{equation}
Applying the boundary conditions to $u_2$, we determine the constants $A$, $B$, $C$, and $D$. The no-slip conditions at the sides ($\hat{z} = 0$ and $\hat{z} = 1$) lead to:
\begin{equation}
    C = 0, \quad \cos\left(\lambda \right) = 0 \implies \lambda_n = \frac{(2n - 1) \pi}{2}
\end{equation}
Thus, the homogeneous solution simplifies to:
\begin{equation}
    u_2(\hat{y}, \hat{z}) = \sum_{n=1}^{\infty} E_n \sinh\left( \lambda_n \hat{y} \right) \sin\left( \lambda_n \frac{b}{h} \hat{z} \right)
\end{equation}
where $\lambda_n = \frac{n \pi h}{b}$ and $E_n$ are constants determined by the boundary conditions.
Combining the particular and homogeneous solutions, the complete velocity profile is:
\begin{equation}
    \hat{u}(\hat{y}, \hat{z}) = \hat{y} - \frac{\hat{y}^2}{2} + \sum_{n=1}^{\infty} E_n \sinh\left( \lambda_n \hat{y} \right) \sin\left( \lambda_n \frac{b}{
h} \hat{z} \right)
\end{equation}
To find the volumetric flow rate $Q$, we integrate the velocity profile over the cross-sectional area:
\begin{equation}
    Q = \int_0^b \int_0^h u(y, z) \, dy \, dz
\end{equation}
Substituting the dimensional form of $\hat{u}$ and performing the integration, we obtain:
\begin{equation}
    Q = \frac{g \sin \theta h^3 b}{\nu} \left( \frac{1}{3} + \sum_{n=1}^{\infty} \frac{E_n}{\lambda_n^2} \left( 1 - \frac{\sinh(\lambda_n)}{\lambda_n \cosh(\lambda_n)} \right) \right)
\end{equation}
Finally, we compare this flow rate with that predicted by Manning's equation to evaluate its accuracy under laminar flow conditions.
\section*{Conclusions}
Put conclusions here.

\section*{Acknowledgements}
Put acknowledgements here.

% The bibliography is automatically generated from template.bib. Rename template.bib to match this file's name, if you change it. This will require an installation of Biber, which you should not install by hand; you run Latex on this document, then Biber on the shared name (e.g. "template"), then Latex again.
% The names you give in the template.bib file become the names you can cite in this document.
\printbibliography

\section*{Appendix}
Put appendix here.
\end{document}
